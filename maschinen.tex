\chapter{Maschinen}

\newpage

\section{DC-Maschine}

\begin{figure}[h!]
\centering
\includegraphics[scale=\schscale]{dc-motor.pdf}
\caption{Ersatzschaltbild einer DC-Maschine}
\label{sch:dc-maschine}
\end{figure}

\subsection{Verhalten und Kennlinie}

Eine DC- oder Gleichstrommaschine hat grundsätzlich drei Betriebszustände:
\begin{itemize}
	\item Ankerstellbereich
	\item Normbereich
	\item Feldschwächungsbereich
\end{itemize}

\begin{figure}[h!]
\centering
\includegraphics[scale=\schscale]{dc-motor-plot1.pdf}
\includegraphics[scale=\schscale]{dc-motor-plot2.pdf}
\caption{Kennlinie einer DC-Maschine}
\label{fig:dc-motor-kennlinie}
\end{figure}

\noindent
Wichtig sind die foldengen Charakteristiken einer DC-Maschine:

\begin{itemize}
	\item Die Drehzahl ist proportional zur Spannung (falls unbelastet).
	\item Nimmt man der DC-Maschine Moment ab (d.h. belasten), so sinkt
		die Drehzahl linear ab bis hin zum maximalen Strom.
	\item	Blockiert man die Rotation (Reibung, Last) so überhitzt
		der Motor.
\end{itemize}

\subsection{Dynamischer Betrieb}
\[ \begin{array}{l}
U_a = U_i + (R_a \cdot I_a) + \left(L_a \cdot \frac{d I_a}{d t}\right) \\\\
U_e = (R_e \cdot I_e) + \left(L_e \cdot \frac{d I_e}{d t}\right) \\\\
U_i = c \cdot \Phi \cdot \omega_m \\\\
M_{el} = c \cdot \Phi \cdot I_a \\\\
M_{el} = M_{Welle} + M_{Reibung} + \left( J \cdot \frac{d \omega_m}{d t} \right) \\\\
\Phi = \frac{L_e}{N_e} \cdot I_e
\end{array} \]

\subsection{Stationärer Betrieb}
\[ \begin{array}{l}
U_a = U_i + (R_a \cdot I_a) \\\\
U_a = (c \cdot \phi \cdot \omega_m) + (R_a \cdot I_a) \\\\
\omega_m 
	= \frac{U_a - (R_a \cdot I_a)}{c \cdot \phi}
	= \frac{U_a}{c \cdot \phi} - \frac{R_a}{c \cdot \phi} \cdot I_a
	= \frac{U_a}{c \cdot \phi} - \frac{R_a}{(c \cdot \phi)^2} \cdot M
\end{array} \]

\newpage

\section{Serieerregte DC-Maschine}\label{sec:dc-motor-serie}

\begin{figure}[h!]
\centering
\includegraphics[scale=\schscale]{dc-motor-serie.pdf}
\caption{Ersatzschaltbild der serieerregten DC-Maschine}
\label{sch:dc-maschine-serie}
\end{figure}

\[ I_e = I_a \]
\[ \phi = \frac{L_e}{N_e} \cdot I_e = \frac{L_e}{N_e} \cdot I_a \]
\[ \omega_m = \frac{U_a - (R_a + R_e) \cdot I_a}{c \cdot \frac{L_e}{N_e} \cdot I_a} \]
\[ M_{el} = c \cdot \frac{L_e}{N_e} \cdot I_a^2 \]
\[ U_a = (R_a + R_e) \cdot I_{e,a} + U_i + (L_a + L_e) \cdot \frac{d I}{d t} \]
\[ U_i = c \cdot \frac{L_e}{N_e} \cdot I \cdot \omega_m \]
\[ \omega_m = \frac{U_a - (R_a + R_e) \cdot I}{c \cdot \frac{L_e}{N_e} \cdot I} \]
\[ \omega_m \approx \frac{U_a}{c \cdot \frac{L_e}{N_e} \cdot \sqrt{\frac{N_e \cdot M}{c \cdot L_e}}} 
	= \frac{U_a}{\sqrt{c \cdot \frac{L_e}{N_e}}} \cdot M \]

\newpage

\section{Universalmotor}

Diese elektrische Maschine kann sowohl mit DC als auch mit AC betrieben
werden ohne Veränderungen vorzunehmen, deshalb werden die Doppelnamen 
\textit{Serieerregte Gleichstrommaschine} und 
\textit{Einphasen-Reihenschluss"-motor} benutzt.

Wird die Maschine mit DC betrieben, so gelten die Formeln aus dem Kapitel
\ref{sec:dc-motor-serie}. Wird die Maschine mit AC betrieben, so müssen die
folgenden Formeln benutzt werden.

\[ \underline{U} = (R_a + R_e) \cdot \underline{I} + j \cdot \omega \cdot (L_a + L_e) \cdot \underline{I} + \underline{U_i} \]
\[ \underline{U_i} = c \cdot \phi \cdot \omega_m = c_1 \cdot \underline{I} \cdot \omega_m \]
\[ |\underline{U^2}| = (|U_i| + R \cdot |\underline{I}|)^2 + (\omega \cdot L \cdot |\underline{I}|)^2, \quad U_i = c_1 \cdot |\underline{I}| \cdot \omega_m \]

\begin{figure}[h!]
\centering
\caption{Zeigerdiagramm des Universalmotors bei AC-Betrieb}
\includegraphics[scale=\schscale]{dc-motor-serie-plot1.pdf}
\label{fig:universalmotor-zeiger}
\end{figure}

\newpage
\section{Synchronmaschine}
\[ \omega_{mech} = \omega_{D_1} = \frac{\omega_1}{p} \]
\[ p\text{: Polpaarzahl} \]
\[ \omega_1 = f_1 \cdot 2 \pi \qquad \text{elektrischer Kreis} \]
$\qquad$ Index-Bedeutung: $1$: Statorgrössen $D$: Drehfeld
\[ n = \frac{60 \cdot \omega_{mech}}{2 \pi} \]
Beispiel $f_1 = 50 Hz$: 
\[ \begin{array}{@{}c|c}
p & n    \\
\hline
1 & 3000 \\
2 & 1500 \\
3 & 1000 \\
4 & 750  \\
5 & 500  \\
\end{array} \]

\subsection{Ersatzschaltung}
Die folgenden Berechnungen basieren auf folgenden Vereinfachungen: 
\begin{itemize}
  \item stationärer Betrieb, d.h. $\omega_1 = p \cdot \omega_{mech}$
  \item $R = 0$
  \item symmetrisch $\rightarrow$ einphasiges Ersatzschaltbild
\end{itemize}
%%%%%%%%%%%%%%%%%%%%%%%%%%%%%%%%%%%%%%%%
% Hier Bild einfügen
%%%%%%%%%%%%%%%%%%%%%%%%%%%%%%%%%%%%%%%%
\[ x_d = \underbrace{x_a}_{\text{Hauptreaktanz}} + \underbrace{x_{\delta}}
_{\text{Streureaktanz}}\qquad \text{Reaktanzen} \]
$\underline{U_1}$: Phasenspannung \\
$\underline{U_p}$: Polradspannung

\subsection{Inselbetrieb}
%%%%%%%%%%%%%%%%%%%%%%%%%%%%%%%%%%%%%%%%
% Hier Bild einfügen
%%%%%%%%%%%%%%%%%%%%%%%%%%%%%%%%%%%%%%%%
\[ \underline{U_1} + \Delta \underline{U} = \underline{U_p} \]
\[ -\underline{I_1} \cdot \underline{Z} + j x_d \cdot \underline{I_1} 
= \underline{U_p} \]
Leistungen: $P, Q$: Drehstromleistung
\[ P = 3 \cdot Re(\underline{U_1} \cdot (-\underline{I_1})^2) \]
\[ Q = 3 \cdot Im(\underline{U_1} \cdot (-\underline{I_1})^2) \]

\subsection{Verbundbetrieb}
%%%%%%%%%%%%%%%%%%%%%%%%%%%%%%%%%%%%%%%%
% Hier Bild einfügen
%%%%%%%%%%%%%%%%%%%%%%%%%%%%%%%%%%%%%%%%
\[ \underline{U_1} = \underbrace{\underline{I_1} \cdot j x_d}_{\Delta U} + U_p \]

\subsection{Polradwinkel $\vartheta$}
\[ \vartheta = \varphi U_1 - \varphi_{U_{p_{U_1}}} \]
\[ \begin{array}{@{}lll}
\vartheta > 0 & \rightarrow & \text{Motorbetrieb} \\
\vartheta < 0 & \rightarrow & \text{Generatorbetrieb} \\
\vartheta = 0 & \rightarrow & \text{Leerlauf} \\
\end{array} \]

\subsection{Drehmoment}
\[ M = \frac{3 \cdot p \cdot U_1 \cdot U_p}{\omega_1 \cdot x_d} 
\cdot \sin{\vartheta} \]

\subsection{Mechanische Leistung}
\[ P_{mech} = \omega_{mech} \cdot M \]

\section{Asynchronmaschine}
Stator: 
\[ \omega_1 = 2 \pi \cdot f_1 \]
\[ \omega_{D_1} = \frac{\omega_1}{p} \qquad p\text{: Polpaarzahl} \]
Synchrone Drehzahl: 
\[ n_{syn} = \frac{60 \cdot \omega_{D_1}}{2 \pi} = \frac{60 \cdot f_1}{p} \]
Rotor: 
\[ \omega_{mech} = \frac{2 \pi \cdot n}{60} \]
\[ \omega_{D_2} = \omega_{D_1} - \omega_{mech} = \frac{\omega_{D_2}}{p} 
\qquad n\text{: Rotordrehzahl} \]
Schlupf: 
\[ S = \frac{n_{syn} - n}{n_{syn}} = \frac{\omega_2}{\omega_1} 
\qquad \text{$\omega_2$: el. Kreisfrequenz der Rotorgrössen $U_2, \varphi_2$} \]
Im Rotor induzierte Spannung: 
\[ \begin{array}{@{}l@{}l@{}l@{}}
U_{n_1} = N_1 \cdot \frac{d \phi_n}{d t} & \Rightarrow 
& U_{n_1} = \omega_1 \cdot N_1 \cdot \phi_n \\
&& U_{n_2} = \omega_2 \cdot N_2 \cdot \phi_n \\
\end{array} \]
\[ ü = \frac{N_1}{N_2} \]
\[ U_{n_2} = N_2 \cdot \omega_2 \cdot \frac{U_{n_1}}{N_1 \cdot \omega_1} 
= \frac{S \cdot U_{n_1}}{ü} \]
\[ \begin{array}{@{}lll}
S < 0     & \rightarrow & \text{Übersynchron} \\
S = 0     & \rightarrow & \text{Synchron} \\
0 < S < 1 & \rightarrow & \text{Untersynchron} \\
S = 1     & \rightarrow & \text{Stillstand} \\
S > 1     & \rightarrow & \text{Gegenlauf} \\
\end{array} \]

\subsection{Leistungen}
(ideales Modell, kein magn. Strom, streuungsfrei)\\
Stator: 
\[ P_1 = P_\delta = 3 \cdot U_{n_1} \cdot I_1 \]
Luftspalt: 
\[ P_\delta = P_1 - \underbrace{3 \cdot R_1 \cdot {I_1}^2 - P_{Fe}}_
{\text{Falls Statorverlust}} \]
Rotor: 
\[ P_2 = 3 \cdot U_{n_2} \cdot I_2 
= 3 \cdot S \cdot U_{n_1} \cdot I_1 = S \cdot P_\delta \]
\[ \begin{array}{ccccc}
P_1 & \Rightarrow &  P_\delta & \Rightarrow & P_{mech} \\
\Downarrow & & \Downarrow & & \\
3 \cdot R_1 \cdot {I_1}^2 + P_{Fe} & & P_2 & \\
\end{array} \]

\subsection{Betriebsarten}
Motor: 
\[ \begin{array}{llllll}
0 < S < 1 
  & 0 < n < n_{syn} 
  & P_\delta > 0 
  \\ 0 < P_{mech} < P_1 
  & 0 < P_2 < P_\delta 
  & \eta = \frac{P_{mech}}{P_1} < 1 - S
\end{array} \]
Generator: 
\[ \begin{array}{llllll}
S < 0 
  & n > n_{syn} 
  & P_\delta < 0 
  \\ P_{mech} < 0 
  & P_2 > 0
  & \eta = \frac{P_1}{P_{mech}} < \frac{1}{1 - S}
\end{array} \]
Gegenlauf: 
\[ \begin{array}{llllll}
S > 1 
  & n < 0
  & P_\delta > 0 
  & P_{mech} > 0
  & P_2 > 0
\end{array} \]
Induzierte Spannung im Rotor
\[ U_{n_2} = \frac{S}{ü} \cdot U_{n_1} = S \cdot \underbrace{U_{n_{2_0}}}_
{\text{Stillstandsspannung}} \]
\[ S = \frac{\omega_2}{\omega_1} \]
\[ u = \frac{N_1}{N_2} \]
\[ I_2 = \frac{S \cdot U_{n_{2_0}}}{R_2} = U_{n_{2_0}} \cdot 
\frac{1}{\underbrace{\frac{R_2}{S}}_{\text{Schlupfabhängiger Widerstand}}} \]
Aufspaltung von $\frac{R_2}{S}$\\
$R_2$: Rotorwiderstand (ohmscher Widerstand)\\
$R_s$: Laufwiderstand (Modell mechanische Energiewandlung)
\[ 3 \cdot {I_2}^2 \cdot R_s \hat{=} P_{mech} 
\qquad \begin{array}{l}>0,~\text{Motor}\\<0,~\text{Generator}\end{array} \]
\[ R_2 + R_s = \frac{R_2}{S} \rightarrow R_s = \frac{R_2 \cdot (1 - S)}{S} \]

\subsection{Drehmoment}
\[ M_{mech} = \frac{P_{mech}}{\omega_{mech}} 
= \frac{3 \cdot {I_2}^2 \cdot R_s}{\omega_{mech}} \]
\[ P_{mech} = \frac{3 \cdot P}{\omega_1} \cdot 
\frac{{U_1}^2}{\left(R_1 + \frac{R_2}{S}\right)^2 + 
(\omega_1 \cdot L_{1_\sigma} + \omega_2 \cdot L_{2_\sigma})^2} \cdot 
\frac{{R_2}^2}{S} \]

\newpage
\section{Transformator}

\subsection{Idealer Transformator}
\begin{itemize}
  \item Verlustlos: \\
        $R_1, R_2 = 0$ \\
        $P_{Fe} = \text{Eisenverluste} = 0$
  \item Streuungsfrei: \\
        $\phi_1 = \phi_2 = \phi$
  \item Kein Magnetisierungsstrom: \\
        $L_n = \infty$
\end{itemize}
\[ U_1 = N_1 \cdot \frac{d \phi}{d t} \qquad U_2 = N_2 \cdot \frac{d \phi}{d t} \]
\[ \frac{U_1}{U_2} = \frac{N_1}{N_2} = \frac{1}{ü} 
\qquad ü:\text{ Übersetzungsverhältnis} \]
\[ R_2' = ü^2 \cdot R_2 \]

\subsection{Realer Transformator}
\begin{itemize}
  \item Verluste: \\
        Wicklungswiderstand $R_1, R_2$ \\
        Eisenverluste: 
        (Wirbelstromverluste, Hystereseverluste (Fläche der Hysterese)) \\
        $R_{Fe} = f(U, f, \dots), $ Typ $P_{Fe} = 1 \dots 10 \frac{W}{kg}$
  \item Streuung: \\
        $\phi_1 \neq \phi_2$ (Flussanteile ohne gegenseitige Kopplung)
  \item Magnetisierungsstrom: \\
        $L_{n_1} 
        = \mu_0 \cdot \mu_r \cdot \frac{A_{Fe}}{\ell_{Fe}} \cdot {N_1}^2$
\end{itemize}

\newpage
\subsection{T-Ersatzschaltbild}
\begin{figure}[h!]
\centering
\includegraphics[scale=\schscale]{trafo-t-ersatz.pdf}
% \caption{T-Ersatzschaltbild eines Transformators}
\label{sch:trafo-t-ersatz}
\end{figure}
\[ R_2' = R_2 \cdot ü^2 \]
\[ X_{\delta_2}' = \cdot X_{\delta_2} \cdot ü^2 \]
\[ U_2' = U_2 \cdot ü \]
\[ I_2' = \frac{I_2}{ü} \]

\subsection{Parameterbestimmung}
Kurzschluss: 
\begin{figure}[h!]
\centering
\includegraphics[scale=\schscale]{trafo-param-short.pdf}
% \caption{Paramaterbestimmung bei Kurzschluss}
\label{sch:trafo-param-short}
\end{figure}
\[ R_1' = R_1 + R_2 \cdot ü^2 \]
\[ X_{\delta_1}' = X_{\delta_1} + X_{\delta_2} \cdot ü^2 \]
\[ U_{K_r} = \frac{U_K}{U_{1_N}} \cdot 100 \% 
\qquad \text{relative Kurzschlussspannung} \]
\[ U_K << U_{1_N} \qquad X_{U_1}, R_{Fe} \text{ vernachlässigt} \]
Leerlauf: 
\begin{figure}[h!]
\centering
\includegraphics[scale=\schscale]{trafo-param-open.pdf}
% \caption{Paramaterbestimmung im Leerlauf}
\label{sch:trafo-param-open}
\end{figure}

\section{Abwärtssteller}
Vereinfachungen: 
\begin{itemize}
  \item periodischer Betrieb
  \item Schaltfrequenz $f: \qquad\frac{L}{R} \gg T = \frac{1}{f} 
        \qquad \frac{1}{2 \pi \cdot \sqrt{L \cdot C}} \ll f$
  \item keine Verluste
\end{itemize}
Schalter geschlossen: 
\[ U_L = U_1 - U_{C2_{Av}} \]
\[ \frac{d i_z}{d t} = \frac{U_1 - U_{C2_{Av}}}{L} \]
Schalter offen: 
\[ U_L = - U_{C2_{Av}} \]
\[ \frac{d i_z}{d t} = \frac{- U_{C2_{Av}}}{L} \]
\[ a = \frac{T_{ein}}{T} \]
\[ U_{2_{Av}} = a \cdot U_1 \]
\[ I_{1_{Av}} = a \cdot I_{2_{Av}} \qquad (P_1 = P_2) \]
\[ a = \frac{U_{2_{Av}}}{U_{1_{Av}}} \]

\newpage
\section{Aufwärtssteller}
Vereinfachungen: 
\begin{itemize}
  \item periodischer Betrieb
  \item Schaltfrequenz $f: \qquad\frac{L}{R} \gg T = \frac{1}{f} 
        \qquad \frac{1}{2 \pi \cdot \sqrt{L \cdot C}} \ll f$
  \item keine Verluste
\end{itemize}
Schalter geschlossen: 
\[ U_L = U_1 \]
\[ \frac{d i_z}{d t} = \frac{U_1}{L} \]
Schalter offen: 
\[ U_L = U_1 - U_{C2_{Av}} \]
\[ \frac{d i_z}{d t} = \frac{U_1 - U_{C2_{Av}}}{L} \]
\[ a = \frac{T_{ein}}{T} \leq 1 \]
\[ U_{2_{Av}} = U_1 \]
\[ I_{1_{Av}} = I_{2_{Av}} \]
\[ a = 1 - \frac{U_{1_{Av}}}{U_{2_{Av}}} \]

\newpage
\section{Einphasiger Gleichrichter}
Vereinfachungen: 
\begin{itemize}
  \item ideale Diode
\end{itemize}
ungesteuert: 
\[ U_{2_0} = \frac{1}{T} \int\limits_0^T u_2 ~dt = \frac{1}{\pi} 
\int\limits_0^\pi u_2 (\underbrace{\varphi}_{\omega t}) ~ d\varphi 
= \frac{2 \sqrt{2}}{\pi} \cdot U_1 = 0.9 \cdot U_1 \]
gesteuert, Zündwinkel $\alpha$:
\[ U_{2_\alpha} = U_{2_0} \cdot \frac{1 + \cos(\alpha)}{2} \]
ideale Stromglättung mit $L \to \infty$: 
\[ U_{2_\alpha} = \frac{1}{\pi} \int\limits_\alpha^{\pi + \alpha} u_d(\varphi) 
~ d \varphi = U_{2_0} \cdot \cos(\alpha) \]
Leistungen: 
\[ P_{2_\alpha} = I_2 \cdot U_{2_\alpha} 
= I_2 \cdot U_{2_0} \cdot \cos(\alpha) \]
\[ I_{1_{eff}} = I_2 \]
\[ S_1 = U_1 \cdot I_{1_{eff}} 
= \frac{\pi}{2 \sqrt{2}} \cdot U_{2_0} \cdot I_2 = 1.1 \cdot U_2 \cdot I_2 \]
\[ \lambda = \frac{P_1}{S_1} = 0.9 \cdot \cos{\alpha} \]
Steuerkennlinien: 
\[ R_{Last} = \frac{1}{\pi} \int\limits_\alpha^\pi u_2 ~ d(\omega t) 
= U_{2_\alpha} \cdot \frac{1 + \cos(\alpha)}{2} \]
L-Glättung: 
\[ U_{2_\alpha} = U_{2_0} \cdot \cos(\alpha) \]
C-Glättung: 
\[ \begin{array}{@{}l@{}l}
U_{2_\alpha} = U_1 \cdot \sqrt{2} & \qquad (0^\circ < \alpha < 90^\circ) \\\\
U_{2_\alpha} = U_1 \cdot \sqrt{2} \cdot \sin(\alpha) 
& \qquad (90^\circ < \alpha < 180^\circ) \\
\end{array} \]

\subsection{Dreiphasige Brückenschaltung}
\[ U_{2_0} = \frac{1}{\frac{\pi}{3}} 
\int\limits_{\frac{\pi}{2} - \frac{\pi}{6}}^{\frac{\pi}{2} + \frac{\pi}{6}} 
u_1 \cdot \sqrt{2} \cdot \sin(\varphi) ~ d \varphi 
= \frac{3 \sqrt{2}}{\pi} \cdot U_1 = 1.35 \cdot U_1 \]
\[ U_{2_\alpha} = U_{2_0} \cdot \cos(\alpha) \]
Ströme bei Stromglättung ($L \to \infty$)
\[ I_1 = \sqrt{\frac{2}{3}} I_2 \qquad \text{(Effektivwert)} \]
\[ I_{HL} = \sqrt{\frac{1}{3}} I_2 \qquad \text{(Effektivwert)} \]
\[ I_{HL_{Av}} = \frac{1}{3} I_2 \]

\newpage
\section{Verluste im Halbleiter}
Leitender Zustand: 
\[ P_{V_{Leit}} = \frac{1}{T} \int\limits_{0}^{T} u \cdot i ~dt 
= U_D \cdot I_{Av} + r_D \cdot I^2 \]
\begin{tabular}{ll}
$U_D$: & Schleusenspannung \\
$r_D$: & differentieller Widerstand \\
\end{tabular}\\
Schaltverluste: 
\[ E_{ein} = \int\limits_0^{t_{ein}} u \cdot i ~ dt 
\qquad E_{aus} = \int\limits_0^{t_{aus}} u \cdot i ~ dt \]
\[ P_{V_{Schalt}} = (E_{ein} + E_{aus}) \cdot f \]

\newpage
\section{Wechselrichter}
Zweipunkt-Betrieb $\to$ nur $f$ steuerbar\\
Dreipunkt-Betrieb $\to$ $f$ und $\alpha$ steuerbar\\\\
Grundfrequenztaktung: 
\[ \hat{U}_{a_1} = \frac{4}{\pi} U_{DC} \cdot \cos(\alpha) 
\qquad \text{Grundschwingung $\omega_1 = 2 \pi f$} \]
Oberschwingungen: 
\[ \hat{U}_{a_v} = \frac{4}{\pi} U_{DC} \cdot \frac{1}{v} \cdot \cos(\alpha) 
\qquad v = 3, 5, 7, \dots \]

\subsection{Wechselrichter am AC Netz}
Vereinfachte Betrachtung: Nur Grundschwingung 
$U_a = \hat{U}_{a_1} \cdot \sin(\omega t + \varphi)$
\[ P \to Re(U_{1_N} \cdot I^*) \]
\[ Q \to Im(U_{1_N} \cdot I^*) \]
\[ U_{1_N} = \hat{U}_{1_N} \cdot \sin(\omega t) \]
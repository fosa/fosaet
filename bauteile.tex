% coding:utf-8

%FOSAET, a LaTeX-Code for a electrical summary of basic electronics
%Copyright (C) 2013, Daniel Winz, Ervin Mazlagic

%This program is free software; you can redistribute it and/or
%modify it under the terms of the GNU General Public License
%as published by the Free Software Foundation; either version 2
%of the License, or (at your option) any later version.

%This program is distributed in the hope that it will be useful,
%but WITHOUT ANY WARRANTY; without even the implied warranty of
%MERCHANTABILITY or FITNESS FOR A PARTICULAR PURPOSE.  See the
%GNU General Public License for more details.
%----------------------------------------

\chapter{Bauteile}
\newpage

%Widerstand 
\input{wid}             % Widerstand
% coding:utf-8

%FOSAET, a LaTeX-Code for a electrical summary of basic electronics
%Copyright (C) 2013, Daniel Winz, Ervin Mazlagic

%This program is free software; you can redistribute it and/or
%modify it under the terms of the GNU General Public License
%as published by the Free Software Foundation; either version 2
%of the License, or (at your option) any later version.

%This program is distributed in the hope that it will be useful,
%but WITHOUT ANY WARRANTY; without even the implied warranty of
%MERCHANTABILITY or FITNESS FOR A PARTICULAR PURPOSE.  See the
%GNU General Public License for more details.
%----------------------------------------

\subsection{Widerstand einer Leitung}
\[ R = \frac{\rho \cdot \ell}{A} \]
\begin{tabular}{lp{0.8\textwidth}}
$\rho$&Spezifischer Widerstand\\
&(Achtung! liegt meist nicht in SI-Einheiten vor)\\
$\ell$&Länge\\
   $A$&Fläche
\end{tabular}

\subsubsection{Spezifischer Widerstand gängiger Materialien}
\begin{table}[h!]
\begin{tabular}{lr}
  Silber    & $1.63 \cdot 10^{-2} \frac{\Omega \cdot mm^2}{m}$ \\
  Kupfer    & $1.73 \cdot 10^{-2} \frac{\Omega \cdot mm^2}{m}$ \\
  Gold      & $2.21 \cdot 10^{-2} \frac{\Omega \cdot mm^2}{m}$ \\
  Aluminium & $2.63 \cdot 10^{-2} \frac{\Omega \cdot mm^2}{m}$ \\
  Messing   & $7.52 \cdot 10^{-2} \frac{\Omega \cdot mm^2}{m}$ \\
  Manganin  & $0.435 \frac{\Omega \cdot mm^2}{m}$ \\
\end{tabular}
\label{tab_spezwid}
\caption{Werte aus den Unterrichtsunterlagen}
\end{table}        % Spezifischer Widerstand
\input{wid_idens}       % Stromdichte
\input{wid_spann}       % Spannungsabhängigkeit
% coding:utf-8

%FOSAET, a LaTeX-Code for a electrical summary of basic electronics
%Copyright (C) 2013, Daniel Winz, Ervin Mazlagic

%This program is free software; you can redistribute it and/or
%modify it under the terms of the GNU General Public License
%as published by the Free Software Foundation; either version 2
%of the License, or (at your option) any later version.

%This program is distributed in the hope that it will be useful,
%but WITHOUT ANY WARRANTY; without even the implied warranty of
%MERCHANTABILITY or FITNESS FOR A PARTICULAR PURPOSE.  See the
%GNU General Public License for more details.
%----------------------------------------

\subsection{Frequenzabhängigkeit}
\begin{figure}[h!]
  \centering
  \begin{circuitikz}[scale=1]\draw
    (0,0) to[short, o-*] (1,0)
    (1,0) to[short, *-] (1,1)
    (5,0) to[short, *-o] (6,0)
    (5,1) to[short, -*] (5,0)
    (1,0) to[R=R, *-] (3,0)
    (3,0) to[L=L, -*] (5,0)
    (1,1) to[C=C, ] (5,1)
    ;
  \end{circuitikz}
  \caption{Ersatzschaltbild bei hohen Frequenzen}
\end{figure}
\begin{tabular}{@{}lp{0.5\textwidth}}
R: & Widerstand \\
C: & Parallelkapazität ($\sim 1 pF$) \\
L: & Serieinduktivität ($\sim 1 - 10 nH$) \\
   & (Abhängig von der Bauform)
\end{tabular}        % Frequenzabhängigkeit
% coding:utf-8

%FOSAET, a LaTeX-Code for a electrical summary of basic electronics
%Copyright (C) 2013, Daniel Winz, Ervin Mazlagic

%This program is free software; you can redistribute it and/or
%modify it under the terms of the GNU General Public License
%as published by the Free Software Foundation; either version 2
%of the License, or (at your option) any later version.

%This program is distributed in the hope that it will be useful,
%but WITHOUT ANY WARRANTY; without even the implied warranty of
%MERCHANTABILITY or FITNESS FOR A PARTICULAR PURPOSE.  See the
%GNU General Public License for more details.
%----------------------------------------

\subsection{Rauschen}
Rauschleistung
\[ P_N = \frac{{U_N}^2}{R} = 4 \cdot k_b \cdot \vartheta \cdot \Delta f \]
\begin{tabular}{@{}lp{0.5\textwidth}}
  $P_N$         & Rauschleistung \\
  $U_N$         & Rauschspannung \\
  $k_b$         & Bolzmann-Konstante ($1.38066 [\frac{J}{K}]$) \\
  $\vartheta$   & Absoluttemperatur $[K]$\\
  $\Delta f$    & Bandbreite $[Hz]$ \\
\end{tabular}

\subsubsection{Addition der Rauschleistung}
\[ P_t = P_1 + P_2 + \dots \]
\[ U_t = \sqrt{{U_1}^2 + {U_2}^2 + \dots} \]      % Rauschen
% coding:utf-8

%FOSAET, a LaTeX-Code for a electrical summary of basic electronics
%Copyright (C) 2013, Daniel Winz, Ervin Mazlagic

%This program is free software; you can redistribute it and/or
%modify it under the terms of the GNU General Public License
%as published by the Free Software Foundation; either version 2
%of the License, or (at your option) any later version.

%This program is distributed in the hope that it will be useful,
%but WITHOUT ANY WARRANTY; without even the implied warranty of
%MERCHANTABILITY or FITNESS FOR A PARTICULAR PURPOSE.  See the
%GNU General Public License for more details.
%----------------------------------------

\subsection{Widerstandsreihen / E-Reihen}
\subsubsection{Berechnung}
\[ R_k = {\sqrt[n]{10}}^m \]
\begin{tabular}{@{}ll}
$R_k$: & Widerstandswert \\
$n$:   & Typ der Reihe (E $n$) (z.B. E12 $\rightarrow$ $n=12$) \\
$m$:   & Stelle des Widerstandswertes in der Reihe \\
\end{tabular} \\
\textbf{Achtung!} Die Werte sind nicht korrekt mathematisch gerundet. Sie sind 
aus der Tabelle auf Seite \pageref{subsubsec:ereihe_tab} zu entnehmen. 

\subsection{Toleranz}
\begin{tabular}{ll}
Reihe & Toleranz \\
E3   & $\leq20\%$ \\
E6   & $20\%$ \\
E12  & $10\%$ \\
E24  & $5\%$ \\
E48  & $2\%$ \\
E96  & $1\%$ \\
E192 & $0.5\%$ \\
\end{tabular}

\subsubsection{Tabelle}
\label{subsubsec:ereihe_tab}
\begin{tabular}{c@{ }c@{ }c@{ }c@{ }c}
E3   & E6   & E12  & E24  \\
1,00 & 1,00 & 1,00 & 1,00 \\
     &      &      & 1,10 \\
     &      & 1,20 & 1,20 \\
     &      &      & 1,30 \\
     & 1,50 & 1,50 & 1,50 \\
     &      &      & 1,60 \\
     &      & 1,80 & 1,80 \\
     &      &      & 2,00 \\
\end{tabular}
\begin{tabular}{c@{ }c@{ }c@{ }c@{ }c}
  E3 & E6   & E12  & E24  \\
2,20 & 2,20 & 2,20 & 2,20 \\
     &      &      & 2,40 \\
     &      & 2,70 & 2,70 \\
     &      &      & 3,00 \\
     & 3,30 & 3,30 & 3,30 \\
     &      &      & 3,60 \\
     &      & 3,90 & 3,90 \\
     &      &      & 4,30 \\
\end{tabular}
\begin{tabular}{c@{ }c@{ }c@{ }c@{ }c}
  E3 & E6   & E12  & E24  \\
4,70 & 4,70 & 4,70 & 4,70 \\
     &      &      & 5,10 \\
     &      & 5,60 & 5,60 \\
     &      &      & 6,20 \\
     & 6,80 & 6,80 & 6,80 \\
     &      &      & 7,50 \\
     &      & 8,20 & 8,20 \\
     &      &      & 9,10 \\
\end{tabular}
      % Widerstandsreihen, E-Reihen
% coding:utf-8

%FOSAET, a LaTeX-Code for a electrical summary of basic electronics
%Copyright (C) 2013, Daniel Winz, Ervin Mazlagic

%This program is free software; you can redistribute it and/or
%modify it under the terms of the GNU General Public License
%as published by the Free Software Foundation; either version 2
%of the License, or (at your option) any later version.

%This program is distributed in the hope that it will be useful,
%but WITHOUT ANY WARRANTY; without even the implied warranty of
%MERCHANTABILITY or FITNESS FOR A PARTICULAR PURPOSE.  See the
%GNU General Public License for more details.
%----------------------------------------

\subsection{Temperaturabhängigkeit von Widerständen}

\subsubsection{Lineare Temperaturabhängigkeit von Widerständen}
\[ R_\vartheta = R_{20} \cdot (1 + \alpha \cdot \Delta \vartheta) \]
\[ \Delta R = R_{20} \cdot \alpha \cdot \Delta \vartheta \]
Falls $R_{20}$ nicht bekannt ist, kann mit $R_A$ bei $\vartheta_a$ und $\tau$ 
die Temperaturabhängigkeit mit folgender Formel berechnet werden:  
\[ R_\vartheta = R_A \frac{\tau + \vartheta}{\tau + \vartheta_A} 
\qquad \text{Wobei} 
\qquad \tau = \frac{1}{\alpha_20} - 20^{\circ}\text{C} \]

\subsubsection{Platintemperatursensoren (PT100, PT1000 \dots)}
Zwischen $0^\circ$ und $100^\circ$ kann die Temperaturabhängigkeit linear 
berechnet werden. 
\[ \begin{array}{l}
R_\vartheta = R_0 \cdot (1 + a \cdot \vartheta) \\
a = 3.85 \cdot 10^{-3} \left[\frac{1}{K}\right] 
\end{array} \]
%
Für höhere Temperaturen oder höhere Genauigkeit wird ein Polynom vom Grad 2 
verwendet. 
\[ \begin{array}{l}
R_\vartheta = R_0 \cdot (1 + a \cdot \vartheta + b \cdot \vartheta^2) \\
a = 3.9083 \cdot 10^{-3} \left[\frac{1}{K}\right] \\
b = -5.775 \cdot 10^{-7} \left[\frac{1}{K^2}\right] 
\end{array} \]
%
Für Temperaturen unter $0^\circ C$ wird ein Polynom vom Grad 4 verwendet. 
\[ \begin{array}{l}
R_\vartheta = R_0 \cdot (1 + a \cdot \vartheta + b \cdot \vartheta^2 
+ c \cdot (\vartheta - 100^\circ C) \cdot \vartheta^3) \\
a = 3.9083 \cdot 10^{-3} \left[\frac{1}{K}\right] \\
b = -5.775 \cdot 10^{-7} \left[\frac{1}{K^2}\right] \\
c = -4.183 \cdot 10^{-12} \left[\frac{1}{K^3}\right] 
\end{array} \]


\subsubsection{Nichtlineare Temperaturabhängigkeit (PTC)}
% \[ R_N = 2 \cdot R_A \]   Diese Formel macht nur mit Grafik Sinn!!!
\[ R_T = R_N \cdot e^{\alpha (T - T_N)} \]
\[ \alpha = \frac{\ln R_1 - \ln R_2}{T_2 - T_1} 
= \frac{\ln\left(\frac{R_1}{R_2}\right)}{\Delta T} 
= \frac{d R_t}{d T}\frac{1}{R_T} \]
\begin{tabular}{@{}lp{0.5\textwidth}}
  $t_A$:        & Anfangstemperatur $[^\circ C]$ \\
  $t_N$:        & Nenntemperatur $[^\circ C]$ \\
  $R_N$:        & Anfangswiderstand $[\Omega]$ \\
  $\alpha$:     & Temperaturkoeffizient oberhalb der Nenntemperatur 
                  ($+5\frac{\%}{K}$ bis $+70\frac{\%}{K}$) \\
  $R_1$, $R_2$: & Widerstände $[\Omega]$ für zwei Punkte oberhalb der 
                  Nenntemperatur \\
  $\Delta T$:   & Temperaturunterschied $[^\circ C]$ für die beiden Punkte mit 
                  $R_1$ und $R_2$
\end{tabular}

\subsubsection{Nichtlineare Temperaturabhängigkeit (NTC)}
\[ R_T = R_N \cdot e^{B\left(\frac{1}{T} - \frac{1}{T_N}\right)} 
= A \cdot e^{\frac{B}{T}} \]
\[ \alpha_T = \frac{d R_T}{d T}\frac{1}{R_T} = -\frac{B}{T^2} 
\qquad B = -\alpha_T \cdot T^2 \]
\begin{tabular}{@{}lp{0.5\textwidth}}
  $R_T$:        & Heissleiterwiderstand bei $T[K]$ in $[\Omega]$ \\
  $R_N$:        & R bei Bezugstemperatur 
                  ($293 K$ entsprechen $20^\circ C$ $[\Omega]$) \\
  $T$:          & Betriebs-, Umgebungstemperatur $[K]$ \\
  $T_N$:        & Bezugstemperatur $[K]$ \\
  $A$:          & Formkonstante $[\Omega]$ nach Herstellerangaben \\
  $B$:          & Regelkonstante $[K]$ \\
  $\alpha_T$:   & temperaturabhängiger Temperaturkoeffizient bei $T$ $[K]$
\end{tabular}        % Temperaturabhängigkeit von Widerständen
\input{wid_vdr}         % Varistor
\input{wid_ldr}         % LDR

% Kondensator
\input{kond}            % Kondensator
% coding:utf-8

%FOSAET, a LaTeX-Code for a electrical summary of basic electronics
%Copyright (C) 2013, Daniel Winz, Ervin Mazlagic

%This program is free software; you can redistribute it and/or
%modify it under the terms of the GNU General Public License
%as published by the Free Software Foundation; either version 2
%of the License, or (at your option) any later version.

%This program is distributed in the hope that it will be useful,
%but WITHOUT ANY WARRANTY; without even the implied warranty of
%MERCHANTABILITY or FITNESS FOR A PARTICULAR PURPOSE.  See the
%GNU General Public License for more details.
%----------------------------------------

\subsection{Kapazität}
\[ C = \frac{A \cdot \varepsilon}{\ell} 
= \frac{A \cdot \varepsilon_0 \cdot \varepsilon_r}{\ell} \]
\begin{tabular}{lp{0.8\textwidth}}
$\varepsilon$&Dielektrizitätskonstante\\
$\varepsilon_0$&Dielektrozitätsoknstante von Vakuum\\
$\varepsilon_r$&relative Dielektrizitätskonstante\\
$A$&Plattenfläche\\
$\ell$&Plattenabstand
\end{tabular}        % Kapazität
\input{kond_ener}       % Energie im Kondensator
\input{kond_parser}     % Parallel und Serieschaltung
% coding:utf-8

%FOSAET, a LaTeX-Code for a electrical summary of basic electronics
%Copyright (C) 2013, Daniel Winz, Ervin Mazlagic

%This program is free software; you can redistribute it and/or
%modify it under the terms of the GNU General Public License
%as published by the Free Software Foundation; either version 2
%of the License, or (at your option) any later version.

%This program is distributed in the hope that it will be useful,
%but WITHOUT ANY WARRANTY; without even the implied warranty of
%MERCHANTABILITY or FITNESS FOR A PARTICULAR PURPOSE.  See the
%GNU General Public License for more details.
%----------------------------------------

\newpage
\subsection{Ersatzschaltung}
\begin{figure}[h!]
  \centering
  \begin{circuitikz}[scale=1]\draw
    (0,0) to[short, o-] (1,0)
    (7,0) to[short, -o] (8,0)
    (1,0) to[R=$R_C$, ] (3,0)
    (3,0) to[L=$L_C$, ] (5,0)
    (5,0) to[C=C, ] (7,0)
    ;
  \end{circuitikz}
  \caption{Ersatzschaltung eines Kondensators}
\end{figure}
     % Ersatzschaltung Kondensator
% coding:utf-8

%FOSAET, a LaTeX-Code for a electrical summary of basic electronics
%Copyright (C) 2013, Daniel Winz, Ervin Mazlagic

%This program is free software; you can redistribute it and/or
%modify it under the terms of the GNU General Public License
%as published by the Free Software Foundation; either version 2
%of the License, or (at your option) any later version.

%This program is distributed in the hope that it will be useful,
%but WITHOUT ANY WARRANTY; without even the implied warranty of
%MERCHANTABILITY or FITNESS FOR A PARTICULAR PURPOSE.  See the
%GNU General Public License for more details.
%----------------------------------------

\subsection{Verluste im Kondensator}
\[ Q_C = \frac{P_B}{P_W} = \tan(\varphi) = \frac{1}{\tan(\delta)} 
\qquad \delta = 90^\circ - \varphi \]
\begin{tabular}{@{}ll}
  $Q_C$: & Güte \\
  $P_B$: & Blindleistung \\
  $P_W$: & Wirkleistung
\end{tabular}

\subsubsection{Serieersatzschaltung}
\begin{figure}[h!]
  \centering
  \begin{circuitikz}[scale=1]\draw
    (0,0) to[short, o-] (1,0)
    (5,0) to[short, -o] (6,0)
    (1,0) to[R=$R_s$, ] (3,0)
    (3,0) to[C=$C$, ] (5,0)
    ;
  \end{circuitikz}
  \caption{Serieersatzschaltung}
\end{figure}
\[ \tan(\delta_s) = R_s \cdot \omega \cdot C = \frac{1}{Q_c} \]

\newpage
\subsubsection{Parallelersatzschaltung}
\begin{figure}[h!]
  \centering
  \begin{circuitikz}[scale=1]\draw
    (0,1) to[short, o-*] (1,1)
    (3,1) to[short, *-o] (4,1)
    (1,0) to[short, ] (1,2)
    (3,0) to[short, ] (3,2)
    (1,2) to[C=$C$, ] (3,2)
    (1,0) to[R=$R_p$, ] (3,0)
    ;
  \end{circuitikz}
  \caption{Parallelersatzschaltung}
\end{figure}
\[ \tan(\delta_p) = \frac{1}{R_p \cdot \omega \cdot C} \]
\[ \text{für } \tan(\delta) < 0.1 :
\qquad \tan(\delta) = \tan(\delta_s) = \tan(\delta_p) \]

\subsubsection{Verlustleistung}
\[ P_v = R_s \cdot i^2 \]
\[ P_v = 2 \cdot \pi \cdot f \cdot C \cdot U^2 \cdot \tan(\delta) \]       % Verluste im Kondensator



% Induktivität
% coding:utf-8

%FOSAET, a LaTeX-Code for a electrical summary of basic electronics
%Copyright (C) 2013, Daniel Winz, Ervin Mazlagic

%This program is free software; you can redistribute it and/or
%modify it under the terms of the GNU General Public License
%as published by the Free Software Foundation; either version 2
%of the License, or (at your option) any later version.

%This program is distributed in the hope that it will be useful,
%but WITHOUT ANY WARRANTY; without even the implied warranty of
%MERCHANTABILITY or FITNESS FOR A PARTICULAR PURPOSE.  See the
%GNU General Public License for more details.
%----------------------------------------

\newpage
\section{Spule}
\[ L = \frac{N \cdot \Phi}{I} \]
\begin{tabular}{@{}ll}
  $L$:      & Induktivität, $\left[\frac{Vs}{A} = 1H \text{ (Henry)}\right]$ \\
  $I$:      & Strom $[A]$ \\
  $N$:      & Anzahl Wicklungen \\
  $\Phi$    & Magnetischer Fluss $\left[ Vs = 1Wb \text{ (Weber)}\right]$
\end{tabular}             % Spule
% coding:utf-8

%FOSAET, a LaTeX-Code for a electrical summary of basic electronics
%Copyright (C) 2013, Daniel Winz, Ervin Mazlagic

%This program is free software; you can redistribute it and/or
%modify it under the terms of the GNU General Public License
%as published by the Free Software Foundation; either version 2
%of the License, or (at your option) any later version.

%This program is distributed in the hope that it will be useful,
%but WITHOUT ANY WARRANTY; without even the implied warranty of
%MERCHANTABILITY or FITNESS FOR A PARTICULAR PURPOSE.  See the
%GNU General Public License for more details.
%----------------------------------------

\subsection{Magnetismus}

\subsubsection{Durchflutung}
\[ \Theta = I \cdot N \]

\subsubsection{Magnetische Feldstärke}
\[ H = \frac{\Theta}{\ell} \left[\frac{A}{m}\right] \]

\subsubsection{Magnetische Induktion}
\[ B = \mu_r \cdot \mu_0 \cdot H \left[\frac{Vs}{m^2} = 1T \text{ Tesla}\right] \]

\subsubsection{Magnetischer Fluss}
\[ \Phi = \int_A B \cdot dA \left[\frac{Vs}{m^2} \cdot m^2 = Vs\right] \]
Homogene Felder
\[ \Phi = B \cdot A \]

\subsubsection{Induktivität}
\[ L = \frac{N \cdot \Phi}{I} \]
         % Magnetismus
% coding:utf-8

%FOSAET, a LaTeX-Code for a electrical summary of basic electronics
%Copyright (C) 2013, Daniel Winz, Ervin Mazlagic

%This program is free software; you can redistribute it and/or
%modify it under the terms of the GNU General Public License
%as published by the Free Software Foundation; either version 2
%of the License, or (at your option) any later version.

%This program is distributed in the hope that it will be useful,
%but WITHOUT ANY WARRANTY; without even the implied warranty of
%MERCHANTABILITY or FITNESS FOR A PARTICULAR PURPOSE.  See the
%GNU General Public License for more details.
%----------------------------------------

\subsection{Induktivität verschiedener Spulenformen}

\subsubsection{Ringkernspule}
\[ L = \frac{N^2 \cdot \mu_0 \cdot \mu_r \cdot A}{\ell_m} \]
\[ \mu = \mu_0 \cdot \mu_r \]
\begin{tabular}{@{}ll}
  $\mu$:    & Permeabilität, megnetische Leitfähigkeit eines Stoffes \\
  $\mu_0$:  & magnetische Feldkonstante für Vakuum \\
            & $\mu_0 = 4 \cdot \pi \cdot 10^{-7} \left[\frac{Vs}{Am}\right]$ \\
  $\mu_r$:  & relative Permeabilität, Permeabilitätszahl \\
  $\ell_m$: & mittlere Feldlinienlänge \\
  $A$:      & Kernquerschnittsfläche \\
  $N$:      & Windungszahl
\end{tabular}

\subsubsection{Kreisringspule}
\[ L = \mu \cdot R \cdot \left(\ln\left(\frac{R}{r}\right) + 0.08\right) \]
\[ \mu = \mu_0 \qquad \text{in nichtmagnetischem Material (Luft)} \]
\begin{tabular}{@{}ll}
  $R$:  & Ringradius \\
  $r$:  & Toroidradius 
\end{tabular}

\subsubsection{Zylinderspule}
Verhältnis Länge - Durchmesser : $\ell \sim 5 \cdot D$:
\[ L = \mu_0 \cdot \frac{N^2 \cdot D^2 \cdot \pi}{4 \cdot \ell} \]
$\ell \sim D$:
\[ L = \mu_0 \cdot \frac{N^2 \cdot D^2 \cdot \pi}{4 \cdot (\ell + 0.45 \cdot D)} \]

\subsubsection{Gegebener Kern}
\[ L = A_L \cdot N^2 \qquad \text{$A_L$: Materialkonstante} \]        % Induktivität verschiedener Spulenformen
% coding:utf-8

%FOSAET, a LaTeX-Code for a electrical summary of basic electronics
%Copyright (C) 2013, Daniel Winz, Ervin Mazlagic

%This program is free software; you can redistribute it and/or
%modify it under the terms of the GNU General Public License
%as published by the Free Software Foundation; either version 2
%of the License, or (at your option) any later version.

%This program is distributed in the hope that it will be useful,
%but WITHOUT ANY WARRANTY; without even the implied warranty of
%MERCHANTABILITY or FITNESS FOR A PARTICULAR PURPOSE.  See the
%GNU General Public License for more details.
%----------------------------------------

\subsection{Induktionsgesetz}
Induktion einer Drahtschleife
\[ u_0(t) = -\frac{d \Phi}{d t} \]
statische Induktivität
\[ N \cdot \Phi = L \cdot I \]
dynamischer Induktionsvorgang
\[ u = - N \cdot \frac{d \Phi}{d t} \]
Selbstinduktion
\[ u = - L  \cdot \frac{d i}{d t} \]
\[ u_L = + L  \cdot \frac{d i}{d t} \]   % Induktionsgesetz
\input{ind_ener}        % Energie in der Spule
\input{ind_parser}      % Parallel- und Seriaschaltung
\input{ind_uconst}      % Spule an konstanter Spannung
% coding:utf-8

%FOSAET, a LaTeX-Code for a electrical summary of basic electronics
%Copyright (C) 2013, Daniel Winz, Ervin Mazlagic

%This program is free software; you can redistribute it and/or
%modify it under the terms of the GNU General Public License
%as published by the Free Software Foundation; either version 2
%of the License, or (at your option) any later version.

%This program is distributed in the hope that it will be useful,
%but WITHOUT ANY WARRANTY; without even the implied warranty of
%MERCHANTABILITY or FITNESS FOR A PARTICULAR PURPOSE.  See the
%GNU General Public License for more details.
%----------------------------------------

\subsection{Ersatzschaltung}
\subsubsection{physikalisches Ersatzschaltbild}
\begin{figure}[h!]
  \centering
  \begin{circuitikz}[scale=1]\draw
    (0,0) to[short, o-] (1,0)
    (5,0) to[short, -o] (6,0)
    (3,0) to[short, *-] (3,1)
    (5,0) to[short, *-] (5,1)
    (1,0) to[R=$R_{Cu}$, ] (3,0)
    (3,1) to[R=$R_K$, ] (5,1)
    (3,0) to[L=$L$, ] (5,0)
    ;
  \end{circuitikz}
  \caption{Ersatzschaltung einer Spule}
\end{figure}
\[ \tan(\delta_{Cu}) = \frac{R_{Cu}}{\omega \cdot L} \]
\[ \tan(\delta_K) = \frac{\omega \cdot L}{R_{Cu}} \]
\begin{tabular}{@{}ll}
  $R_{Cu}$: & Kupferverlustwiderstand \\
  $R_K$:    & Kernverlustwiderstand \\
\end{tabular}

\subsubsection{Serieersatzschaltbild}
\begin{figure}[h!]
  \centering
  \begin{circuitikz}[scale=1]\draw
    (0,0) to[short, o-] (1,0)
    (7,0) to[short, -o] (8,0)
    (1,0) to[R=$R_{Cu}$, ] (3,0)
    (3,0) to[R=$R_{K_s}$, ] (5,0)
    (5,0) to[L=$L$, ] (7,0)
    ;
  \end{circuitikz}
  \caption{Serieersatzschaltung einer Spule}
\end{figure}
\[ Q = \frac{\omega \cdot L}{R_s} = \frac{1}{\tan(\delta)} \]
\[ R_s = R_{Cu} + R_{K_s} \]
\[ R_{K_s} \approx \tan^2(\delta_K) \cdot R_K \]

\newpage
\subsubsection{Parallelersatzschaltbild}
\begin{figure}[h!]
  \centering
  \begin{circuitikz}[scale=1]\draw
    (0,0) to[short, o-] (1,0)
    (3,0) to[short, -o] (4,0)
    (1,0) to[short, *-] (1,1)
    (1,1) to[short, *-] (1,2)
    (3,0) to[short, *-] (3,1)
    (3,1) to[short, *-] (3,2)
    (1,2) to[R=$R_{Cu_p}$, ] (3,2)
    (1,1) to[R=$R_K$, ] (3,1)
    (1,0) to[L=$L$, ] (3,0)
    ;
  \end{circuitikz}
  \caption{Parallelersatzschaltung einer Spule}
\end{figure}
\[ Q = \frac{R_p}{\omega \cdot L} = \frac{1}{\tan(\delta)} \]
\[ R_p = R_K // R_{Cu_p} = \frac{R_K \cdot R_{Cu_p}}{R_K + R_{Cu_p}} \]
\[ R_{Cu_p} = \frac{R_{Cu}}{\tan^2(\delta_{Cu})} \]
      % Ersatzschaltbild Kondensator
\input{ind_loss}        % Verluste in der Spule

% Diode
\input{diode}           % Diode
% coding:utf-8

%FOSAET, a LaTeX-Code for a electrical summary of basic electronics
%Copyright (C) 2013, Daniel Winz, Ervin Mazlagic

%This program is free software; you can redistribute it and/or
%modify it under the terms of the GNU General Public License
%as published by the Free Software Foundation; either version 2
%of the License, or (at your option) any later version.

%This program is distributed in the hope that it will be useful,
%but WITHOUT ANY WARRANTY; without even the implied warranty of
%MERCHANTABILITY or FITNESS FOR A PARTICULAR PURPOSE.  See the
%GNU General Public License for more details.
%----------------------------------------

\subsection{Kennlinie einer Diode}
\[ I_D(v_D, \vartheta) = I_s \cdot \left(e^{\frac{v_d}{n \cdot V_T}} - 1\right) \]
\[ V_T = \frac{k \cdot \vartheta}{q} \]
\[ V_T~(25^\circ C) = 25.8 mV \]
\begin{tabular}{@{}ll}
  $I_s$:        & Sättigungsstrom \\
                & (Diodentyp- und Temperaturabhängig) \\
                & ($400 pA$ für A-, $3 fA$ für mA- Dioden) \\
  $\vartheta$:  & Temperatur [$K$] \\
  $k$:          & Bolzmann-Konstante, $k=1.38 \cdot 10^{-23} \frac{J}{K}$ \\
  $q$:          & Elementarladung, $1.6 \cdot 10^{-19} C$ \\
  $n$:          & Idealitätsfaktor, $n = 1-2$ \\
                & ($n = 1$ typisch, Diodentyp abhängig)
\end{tabular}      % Kennline einer Diode
\input{diode_kleinsig}  % Kleinsignalmodell einer Diode

% Bipolartransistor
\input{trans_grosssig}	% DC_Grosssignalverhalten

% Unipolartransistor

% Operationsverstärker
\input{op}              % Operationsverstärker
% coding:utf-8

%FOSAET, a LaTeX-Code for a electrical summary of basic electronics
%Copyright (C) 2013, Daniel Winz, Ervin Mazlagic

%This program is free software; you can redistribute it and/or
%modify it under the terms of the GNU General Public License
%as published by the Free Software Foundation; either version 2
%of the License, or (at your option) any later version.

%This program is distributed in the hope that it will be useful,
%but WITHOUT ANY WARRANTY; without even the implied warranty of
%MERCHANTABILITY or FITNESS FOR A PARTICULAR PURPOSE.  See the
%GNU General Public License for more details.
%----------------------------------------


\subsection{Frequenzgang - Realer OP}
\[ V_{U_O} \cdot f_{go} = 1 \cdot f_T \]
Wird mit dem Operationsverstärker ein Filter mit Güte realisiert, so 
muss die Güte des Filters berücksichtigt werden. 
\[ V_{U_O} \cdot f_{go} \cdot Q = 1 \cdot f_T \]
\begin{tabular}{@{}ll}
  $V_{U_O}$:    & Openloop Versärkung DC \\
  $f_{go}$:     & Openloop Grenzfrequenz \\   
  $f_T$:        & Transitfrequenz \\
  $Q$:          & Güte \\
\end{tabular}
        % Frequenzgang
\input{op_ausgang}      % Verhalten am Ausgang
\input{op_ufolg}        % Spannungsfolger, Impedanzwandler
% coding:utf-8

%FOSAET, a LaTeX-Code for a electrical summary of basic electronics
%Copyright (C) 2013, Daniel Winz, Ervin Mazlagic

%This program is free software; you can redistribute it and/or
%modify it under the terms of the GNU General Public License
%as published by the Free Software Foundation; either version 2
%of the License, or (at your option) any later version.

%This program is distributed in the hope that it will be useful,
%but WITHOUT ANY WARRANTY; without even the implied warranty of
%MERCHANTABILITY or FITNESS FOR A PARTICULAR PURPOSE.  See the
%GNU General Public License for more details.
%----------------------------------------

\subsection{Nichtinvertierender Verstärker}
\begin{figure}[h!]
	\centering
	\includegraphics[scale=\schscale]{../fig/op_ninv.pdf}
	\caption{Nichtinvertierender Verstärker}
	\label{sch:op-ninv}
\end{figure}
\[ V_u = \frac{R_1}{R_2} + 1 \]
\[ U_a = U_e \cdot \frac{R_1}{R_2} + 1 \]
\[ R_e = \infty \]
\[ R_a = 0 \]         % Nichtinvertierender Verstärker
% coding:utf-8

%FOSAET, a LaTeX-Code for a electrical summary of basic electronics
%Copyright (C) 2013, Daniel Winz, Ervin Mazlagic

%This program is free software; you can redistribute it and/or
%modify it under the terms of the GNU General Public License
%as published by the Free Software Foundation; either version 2
%of the License, or (at your option) any later version.

%This program is distributed in the hope that it will be useful,
%but WITHOUT ANY WARRANTY; without even the implied warranty of
%MERCHANTABILITY or FITNESS FOR A PARTICULAR PURPOSE.  See the
%GNU General Public License for more details.
%----------------------------------------

\subsection{Invertierender Verstärker}
\begin{figure}[h!]
	\centering
	\includegraphics[scale=\schscale]{../fig/op_inv.pdf}
	\caption{Invertierender Verstärker}
	\label{sch:op-inv}
\end{figure}
\[ V_u = - \frac{R_2}{R_1} \]
\[ U_a = - U_e \cdot \frac{R_2}{R_1} \]
\[ R_e = R_1 \qquad R_a = 0 \]          % Invertierender Verstärker
% coding:utf-8

%FOSAET, a LaTeX-Code for a electrical summary of basic electronics
%Copyright (C) 2013, Daniel Winz, Ervin Mazlagic

%This program is free software; you can redistribute it and/or
%modify it under the terms of the GNU General Public License
%as published by the Free Software Foundation; either version 2
%of the License, or (at your option) any later version.

%This program is distributed in the hope that it will be useful,
%but WITHOUT ANY WARRANTY; without even the implied warranty of
%MERCHANTABILITY or FITNESS FOR A PARTICULAR PURPOSE.  See the
%GNU General Public License for more details.
%----------------------------------------

\subsection{Addierer}
\begin{figure}[h!]
	\centering
	\includegraphics[scale=\schscale]{../fig/op_add.pdf}
	\caption{Addierer}
	\label{sch:op-add}
\end{figure}
\[ U_a = - R_2 \cdot \left( \frac{U_{e1}}{R_{11}} + \frac{U_{e2}}{R_{12}} 
+ \frac{U_{e3}}{R_{13}} \right) \]
\[ R_{e1} = R_{11} \qquad R_{e2} = R_{12} \qquad R_{e3} = R_{13} \]
\[ R_a = 0 \]
          % Addierer
% coding:utf-8

%FOSAET, a LaTeX-Code for a electrical summary of basic electronics
%Copyright (C) 2013, Daniel Winz, Ervin Mazlagic

%This program is free software; you can redistribute it and/or
%modify it under the terms of the GNU General Public License
%as published by the Free Software Foundation; either version 2
%of the License, or (at your option) any later version.

%This program is distributed in the hope that it will be useful,
%but WITHOUT ANY WARRANTY; without even the implied warranty of
%MERCHANTABILITY or FITNESS FOR A PARTICULAR PURPOSE.  See the
%GNU General Public License for more details.
%----------------------------------------

\subsection{Subtrahierer / Differenzverstärker}
\begin{figure}[h!]
	\centering
	\includegraphics[scale=\schscale]{op_sub.pdf}
	\caption{Subtrahierer / Differenzverstärker}
	\label{sch:op-sub}
\end{figure}
\[ U_a = \frac{(R_1 + R_2) \cdot R_4}{(R_3 + R_4) \cdot R_1} \cdot U_{e+} 
- \frac{R_2}{R_1} \cdot U_{e-} \]
Wenn $R_1 = R_3$ und $R_2 = R_4$: 
\[ U_a = \frac{R_2}{R_1} \cdot (U_{e+} - U_{e-}) \]
Wenn $R_1 = R_2 = R_3 = R_4$: 
\[ U_a = U_{e+} - U_{e-} \]
\[ R_{e+} = R_3 + R_4 \]
\[ R_{e-} = R_1 \]
\[ R_a = 0 \]          % Subtrahierer
% coding:utf-8

%FOSAET, a LaTeX-Code for a electrical summary of basic electronics
%Copyright (C) 2013, Daniel Winz, Ervin Mazlagic

%This program is free software; you can redistribute it and/or
%modify it under the terms of the GNU General Public License
%as published by the Free Software Foundation; either version 2
%of the License, or (at your option) any later version.

%This program is distributed in the hope that it will be useful,
%but WITHOUT ANY WARRANTY; without even the implied warranty of
%MERCHANTABILITY or FITNESS FOR A PARTICULAR PURPOSE.  See the
%GNU General Public License for more details.
%----------------------------------------

\subsection{Differenzierer}
\begin{figure}[h!]
	\centering
	\includegraphics[scale=\schscale]{../fig/op_diff.pdf}
	\caption{Differenzierer}
	\label{sch:op-diff}
\end{figure}
\[ U_a = - R \cdot C \cdot \frac{d U_e(t)}{dt} \]
\[ X_e = \frac{1}{j \cdot \omega \cdot f \cdot C} \]
\[ R_a = 0 \]         % Differenzierer
% coding:utf-8

%FOSAET, a LaTeX-Code for a electrical summary of basic electronics
%Copyright (C) 2013, Daniel Winz, Ervin Mazlagic

%This program is free software; you can redistribute it and/or
%modify it under the terms of the GNU General Public License
%as published by the Free Software Foundation; either version 2
%of the License, or (at your option) any later version.

%This program is distributed in the hope that it will be useful,
%but WITHOUT ANY WARRANTY; without even the implied warranty of
%MERCHANTABILITY or FITNESS FOR A PARTICULAR PURPOSE.  See the
%GNU General Public License for more details.
%----------------------------------------

\subsection{Integrierer}
\begin{figure}[h!]
	\centering
	\includegraphics[scale=\schscale]{../fig/op_int.pdf}
	\caption{Integrierer}
	\label{sch:op-int}
\end{figure}
\[ U_a = - \frac{\int_{0}^{t} U_e(t) dt}{R \cdot C} + U_a(0) \]
\[ R_e = R \]
\[ R_a = 0 \]          % Integrierer
\input{op_isource}      % Stromquelle
\input{op_iu}           % Strom-Spannungswandler

% Kühlung
% coding:utf-8

%FOSAET, a LaTeX-Code for a electrical summary of basic electronics
%Copyright (C) 2013, Daniel Winz, Ervin Mazlagic

%This program is free software; you can redistribute it and/or
%modify it under the terms of the GNU General Public License
%as published by the Free Software Foundation; either version 2
%of the License, or (at your option) any later version.

%This program is distributed in the hope that it will be useful,
%but WITHOUT ANY WARRANTY; without even the implied warranty of
%MERCHANTABILITY or FITNESS FOR A PARTICULAR PURPOSE.  See the
%GNU General Public License for more details.
%----------------------------------------

\section{Thermisches Ersatzschaltbild}
\begin{figure}[h!]
  \centering
  \begin{circuitikz}[scale=1]\draw
%     (0,0) to[short, o-] (0,0)
%     (0,0) to[short, -o] (0,0)
    (0,6) to[R=$R_{th_{JC}}$, o-o] (0,4)
    (0,4) to[R=$R_{th_{CH}}$, o-o] (0,2)
    (0,2) to[R=$R_{th_{HA}}$, o-o] (0,0)
    (0.4,6) node {J}
    (0.4,4) node {C}
    (0.4,2) node {H}
    (0.4,0) node {A}
    ;
  \end{circuitikz}
  \caption{Thermisches Ersatzschaltbild}
\end{figure}
Äquivalent zu Ohmschem Gesetz: 
\[ \Delta\vartheta = R_{th} \cdot P_v \]
\[ R_{th} = \frac{\ell}{\lambda \cdot A} \]
\begin{tabular}{@{}ll}
  $J$: & Junction $\rightarrow$ Sperrschicht \\
  $C$: & Case $\rightarrow$ Gehäuse \\
  $H$: & Heatsink $\rightarrow$ Kühlkörper \\
  $A$: & Ambient $\rightarrow$ Umgebung (Achtung! Meist innerhalb Gerät) \\
  $\lambda$: & Thermische Leitfähigkeit
\end{tabular}
         % Thermisches Ersatzschaltbild
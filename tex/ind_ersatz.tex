% coding:utf-8

%FOSAET, a LaTeX-Code for a electrical summary of basic electronics
%Copyright (C) 2013, Daniel Winz, Ervin Mazlagic

%This program is free software; you can redistribute it and/or
%modify it under the terms of the GNU General Public License
%as published by the Free Software Foundation; either version 2
%of the License, or (at your option) any later version.

%This program is distributed in the hope that it will be useful,
%but WITHOUT ANY WARRANTY; without even the implied warranty of
%MERCHANTABILITY or FITNESS FOR A PARTICULAR PURPOSE.  See the
%GNU General Public License for more details.
%----------------------------------------

\subsection{Ersatzschaltung}
\subsubsection{physikalisches Ersatzschaltbild}
\begin{figure}[h!]
  \centering
  \begin{circuitikz}[scale=1]\draw
    (0,0) to[short, o-] (1,0)
    (5,0) to[short, -o] (6,0)
    (3,0) to[short, *-] (3,1)
    (5,0) to[short, *-] (5,1)
    (1,0) to[R=$R_{Cu}$, ] (3,0)
    (3,1) to[R=$R_K$, ] (5,1)
    (3,0) to[L=$L$, ] (5,0)
    ;
  \end{circuitikz}
  \caption{Ersatzschaltung einer Spule}
\end{figure}
\[ \tan(\delta_{Cu}) = \frac{R_{Cu}}{\omega \cdot L} \]
\[ \tan(\delta_K) = \frac{\omega \cdot L}{R_{Cu}} \]
\begin{tabular}{@{}ll}
  $R_{Cu}$: & Kupferverlustwiderstand \\
  $R_K$:    & Kernverlustwiderstand \\
\end{tabular}

\subsubsection{Serieersatzschaltbild}
\begin{figure}[h!]
  \centering
  \begin{circuitikz}[scale=1]\draw
    (0,0) to[short, o-] (1,0)
    (7,0) to[short, -o] (8,0)
    (1,0) to[R=$R_{Cu}$, ] (3,0)
    (3,0) to[R=$R_{K_s}$, ] (5,0)
    (5,0) to[L=$L$, ] (7,0)
    ;
  \end{circuitikz}
  \caption{Serieersatzschaltung einer Spule}
\end{figure}
\[ Q = \frac{\omega \cdot L}{R_s} = \frac{1}{\tan(\delta)} \]
\[ R_s = R_{Cu} + R_{K_s} \]
\[ R_{K_s} \approx \tan^2(\delta_K) \cdot R_K \]

\newpage
\subsubsection{Parallelersatzschaltbild}
\begin{figure}[h!]
  \centering
  \begin{circuitikz}[scale=1]\draw
    (0,0) to[short, o-] (1,0)
    (3,0) to[short, -o] (4,0)
    (1,0) to[short, *-] (1,1)
    (1,1) to[short, *-] (1,2)
    (3,0) to[short, *-] (3,1)
    (3,1) to[short, *-] (3,2)
    (1,2) to[R=$R_{Cu_p}$, ] (3,2)
    (1,1) to[R=$R_K$, ] (3,1)
    (1,0) to[L=$L$, ] (3,0)
    ;
  \end{circuitikz}
  \caption{Parallelersatzschaltung einer Spule}
\end{figure}
\[ Q = \frac{R_p}{\omega \cdot L} = \frac{1}{\tan(\delta)} \]
\[ R_p = R_K // R_{Cu_p} = \frac{R_K \cdot R_{Cu_p}}{R_K + R_{Cu_p}} \]
\[ R_{Cu_p} = \frac{R_{Cu}}{\tan^2(\delta_{Cu})} \]
